%%%%%%%%%%%%%%%%%%%%%%%%%%%%%%%%%%%%%%%%%%%%%%%%%%%%%%%%%%%%%%%%%%%%%%
% How to use writeLaTeX: 
%
% You edit the source code here on the left, and the preview on the
% right shows you the result within a few seconds.
%
% Bookmark this page and share the URL with your co-authors. They can
% edit at the same time!
%
% You can upload figures, bibliographies, custom classes and
% styles using the files menu.
%
%%%%%%%%%%%%%%%%%%%%%%%%%%%%%%%%%%%%%%%%%%%%%%%%%%%%%%%%%%%%%%%%%%%%%%

\documentclass[12pt]{article}

\usepackage{sbc-template}

\usepackage{graphicx,url}

%\usepackage[brazil]{babel}   
\usepackage[utf8]{inputenc}  

     
\sloppy

\title{State Machine Replication in Containers Managed by Kubernetes}

\author{Joao Victor Oliveira Farias}


\address{Universidade Federal do Ceará - Campus Quixadá\\
  Quixadá -- CE -- Brasil
\email{victorfarias.new@gmail.com}}
\begin{document} 

\maketitle

\begin{abstract}
  
\end{abstract}
     
\begin{resumo} 
  Este meta-artigo se caracteriza por um breve resumo explicativo sobre o artigo cujo tema é exposto no título do documento. Contêineres possuem grande capacidade de escalabilidade horizontal, e isso decorre principalmente de sua ferramenta de gerenciamento mais comumente usada, que é o Kubernetes. Esta ferramenta possibilita o provisionamento de recursos e replicação de contêineres baseado na demanda das aplicações que utilizam o serviço, criando balanceamento de carga tanto para o acesso aos recursos, quanto para os serviços de rede. O artigo explicita um dos problemas em contêineres, que é a replicação dos seus estados atuais, pois quando um contêiner novo é instanciado ou duplicado, seu estado é totalmente perdido, espelhando as configurações e aplicações do contêiner espelhado, mas sem replicar o estado em que o outro estava. No artigo, o autor propõe uma arquitetura para a integração de um protocolo de replicação de estados, além de propor o protocolo em si. A arquitetura é montada de maneira que os estados dos contêineres são gerenciados através do módulo de memória compartilhada do Kubernetes, o \textit{etcd}, esse módulo tem acesso a memória compartilhada de todos os containers do cluster, tornando possível a implementação de um protocolo que faça o gerenciamento dos nós da arquitetura. Logo em seguida, o autor também propõe a criação do protocolo de controle de replicação, onde cria a hipótese de que os contêineres necessitam de \textit{k + 1} réplicas para garantir que em caso de falhas, sempre haja um substituto para sua execução. Com a criação dessa hipótese, é criado um algoritmo de eleição de líderes dentre os contêineres que precisam ser replicados, a fim de que esse líder só responda a requisição quando o estado dele for replicado ao menos \textit{k + 1} vezes. A partir desse cenário, ele inicia os testes e valida a replicação dos estados, contribuindo de forma a criar uma proposta de integração a atual arquitetura do Kubernetes.\cite{netto2017state}
\end{resumo}


\bibliographystyle{sbc}
\bibliography{sbc-template}

\end{document}
